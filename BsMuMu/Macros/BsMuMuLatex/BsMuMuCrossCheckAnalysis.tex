\documentclass[10pt,a4paper]{article}
\usepackage[latin1]{inputenc}
\usepackage{amsmath}
\usepackage{amsfonts}
\usepackage{amssymb}
\usepackage{graphicx}
\usepackage{caption}
\usepackage{subcaption}
\usepackage{rotating}

\newcommand{\BsMuMu}{B$^{0}_{s}\rightarrow \mu^+\mu^-\ $}
\newcommand{\BMuMu}{B$^{0}s\rightarrow \mu^+\mu^-\ $}

\author{M. De Mattia}
\begin{document}
\title{\BsMuMu Cross Check Analysis}
\maketitle

\begin{abstract}
The aim of this analysis note is to provide supporting material for the cross check analysis for the
search for \BsMuMu and \BMuMu decays. It follows closely the documentation provided in \cite{MainAN}
and includes direct comparison to the results in that note.
\end{abstract}

\section{Datasets}

The datasets used are shown in table \ref{tab:datasets}.
\begin{table}
\begin{tabular}{|l|}
\hline
Official MC datasets \\
\hline
BsToMuMu\_BsFilter\_8TeV-pythia6-evtgen/Summer12\_DR53X-PU\_S10\_START53\_V7A-v1/AODSIM \\
\hline
Data \\
/MuOnia/Run2012A-13Jul2012-v1/AOD \\
/MuOnia/Run2012A-recover-06Aug2012-v1/AOD \\
/MuOnia/Run2012B-13Jul2012-v1/AOD \\
/MuOnia/Run2012C-24Aug2012-v1/AOD \\
/MuOnia/Run2012C-PromptReco-v2/AOD \\
/MuOnia/Run2012C-EcalRecover\_11Dec2012-v1/AOD \\
/MuOnia/Run2012D-PromptReco-v1/AOD \\
\hline
\end{tabular}
\label{tab:datasets}
\end{table}

The events are selected using the same triggers as described in the reference \cite{MainAN}.

The analysis is based on a boosted decision tree (BDT) algorithm to select \BsMuMu events from the
background.
A preselection is applied to the data and MC samples to reduce the size and remove outliers that
might confuse the BDT training.
The signal is taken from the \BsMuMu MC sample, while the background is taken from the sidebands
in the data sample.
\\
Despite the availability of the full 2012 dataset, in order to allow a direct comparison with
the results in \cite{MainAN} this study is limited to runs $<= 203002$.
The dataset is split in three different categories according the remainder of "event number$\%3$".

\section{Selection}

The same preselection as in \cite{MainAN} is used. The muons are chosen to pass global muon prompt tight
selection (GM\_PT).

The distributions of the variables used for the BDT training are shown in Figures \ref{fig:TMVAPlotsBarrel} and \ref{fig:TMVAPlotsEndcaps}.

%add desired spacing between images, e. g. ~, \quad, \qquad etc.
%(or a blank line to force the subfigure onto a new line)
\begin{sidewaysfigure}
        \centering
        \begin{subfigure}[b]{0.2\textwidth}
                \centering
                \includegraphics[width=\textwidth]{Figures/pt_barrel}
                \label{fig:ptBarrel}
        \end{subfigure}
        ~
        \begin{subfigure}[b]{0.2\textwidth}
                \centering
                \includegraphics[width=\textwidth]{Figures/y_barrel}
                \label{fig:yBarrel}
                % \includegraphics[width=\textwidth]{Figures/eta_barrel}
                % \label{fig:etaBarrel}
        \end{subfigure}
        ~
        \begin{subfigure}[b]{0.2\textwidth}
                \centering
                \includegraphics[width=\textwidth]{Figures/fls3d_barrel}
                \label{fig:fls3dBarrel}
        \end{subfigure}
        ~
        \begin{subfigure}[b]{0.2\textwidth}
                \centering
                \includegraphics[width=\textwidth]{Figures/chi2dof_barrel}
                \label{fig:chi2dofBarrel}
        \end{subfigure}
        
        \begin{subfigure}[b]{0.2\textwidth}
                \centering
                \includegraphics[width=\textwidth]{Figures/alpha_barrel}
                \label{fig:alphaBarrel}
        \end{subfigure}
        ~
        \begin{subfigure}[b]{0.2\textwidth}
                \centering
                \includegraphics[width=\textwidth]{Figures/iso_barrel}
                \label{fig:isoBarrel}
        \end{subfigure}
        ~
        \begin{subfigure}[b]{0.2\textwidth}
                \centering
                \includegraphics[width=\textwidth]{Figures/closetrk_barrel}
                \label{fig:closetrkBarrel}
        \end{subfigure}
        ~
        \begin{subfigure}[b]{0.2\textwidth}
                \centering
                \includegraphics[width=\textwidth]{Figures/docatrk_barrel}
                \label{fig:docatrkBarrel}
        \end{subfigure}

        \begin{subfigure}[b]{0.2\textwidth}
                \centering
                \includegraphics[width=\textwidth]{Figures/pvip_barrel}
                \label{fig:pvipBarrel}
        \end{subfigure}
        ~
        \begin{subfigure}[b]{0.2\textwidth}
                \centering
                \includegraphics[width=\textwidth]{Figures/pvips_barrel}
                \label{fig:pvipsBarrel}
        \end{subfigure}
        ~
        \begin{subfigure}[b]{0.2\textwidth}
                \centering
                \includegraphics[width=\textwidth]{Figures/maxdoca_barrel}
                \label{fig:maxdocaBarrel}
        \end{subfigure}
        \caption{Standard TMVA plot of the input variables for the barrel BDT for signal (blue) and background (red). The background is extracted from data dimuon sidebands.}
        \label{fig:TMVAPlotsBarrel}
\end{sidewaysfigure}

\begin{sidewaysfigure}
        \centering
        \begin{subfigure}[b]{0.2\textwidth}
                \centering
                \includegraphics[width=\textwidth]{Figures/pt_endcaps}
                \label{fig:ptEndcaps}
        \end{subfigure}
        ~
        \begin{subfigure}[b]{0.2\textwidth}
                \centering
                \includegraphics[width=\textwidth]{Figures/y_endcaps}
                \label{fig:yEndcaps}
                % \includegraphics[width=\textwidth]{Figures/eta_endcaps}
                % \label{fig:etaEndcaps}
        \end{subfigure}
        ~
        \begin{subfigure}[b]{0.2\textwidth}
                \centering
                \includegraphics[width=\textwidth]{Figures/fls3d_endcaps}
                \label{fig:fls3dEndcaps}
        \end{subfigure}
        ~
        \begin{subfigure}[b]{0.2\textwidth}
                \centering
                \includegraphics[width=\textwidth]{Figures/chi2dof_endcaps}
                \label{fig:chi2dofEndcaps}
        \end{subfigure}
        
        \begin{subfigure}[b]{0.2\textwidth}
                \centering
                \includegraphics[width=\textwidth]{Figures/alpha_endcaps}
                \label{fig:alphaEndcaps}
        \end{subfigure}
        ~
        \begin{subfigure}[b]{0.2\textwidth}
                \centering
                \includegraphics[width=\textwidth]{Figures/iso_endcaps}
                \label{fig:isoEndcaps}
        \end{subfigure}
        ~
        \begin{subfigure}[b]{0.2\textwidth}
                \centering
                \includegraphics[width=\textwidth]{Figures/closetrk_endcaps}
                \label{fig:closetrkEndcaps}
        \end{subfigure}
        ~
        \begin{subfigure}[b]{0.2\textwidth}
                \centering
                \includegraphics[width=\textwidth]{Figures/docatrk_endcaps}
                \label{fig:docatrkEndcaps}
        \end{subfigure}

        \begin{subfigure}[b]{0.2\textwidth}
                \centering
                \includegraphics[width=\textwidth]{Figures/pvip_endcaps}
                \label{fig:pvipEndcaps}
        \end{subfigure}%
        ~
        \begin{subfigure}[b]{0.2\textwidth}
                \centering
                \includegraphics[width=\textwidth]{Figures/pvips_endcaps}
                \label{fig:pvipsEncaps}
        \end{subfigure}
        ~
        \begin{subfigure}[b]{0.2\textwidth}
                \centering
                \includegraphics[width=\textwidth]{Figures/maxdoca_endcaps}
                \label{fig:maxdocaEndcaps}
        \end{subfigure}
        \caption{Standard TMVA plot of the input variables for the endcaps BDT for signal (blue) and background (red). The background is extracted from data dimuon sidebands.}
        \label{fig:TMVAPlotsEndcaps}
\end{sidewaysfigure}

\begin{figure}
		\centering
        \begin{subfigure}[b]{0.45\textwidth}
				\includegraphics[width=\textwidth]{Figures/correlationMatrixS_barrel}
				\label{fig:matrixSBarrel}
        \end{subfigure}
        ~
        \begin{subfigure}[b]{0.45\textwidth}
				\includegraphics[width=\textwidth]{Figures/correlationMatrixS_endcaps}
				\label{fig:matrixSEndcaps}
        \end{subfigure}
        \caption{Correlation matrix for signal events in the barrel (left) and the endcap (right).}
        \label{fig:correlationMatricesSignal}
\end{figure}

\begin{figure}
		\centering
        \begin{subfigure}[b]{0.45\textwidth}
				\includegraphics[width=\textwidth]{Figures/correlationMatrixB_barrel}
				\label{fig:matrixBBarrel}
        \end{subfigure}
        ~
        \begin{subfigure}[b]{0.45\textwidth}
				\includegraphics[width=\textwidth]{Figures/correlationMatrixB_endcaps}
				\label{fig:matrixBEndcaps}
        \end{subfigure}
        \caption{Correlation matrix for background events in the barrel (left) and the endcap (right).}
        \label{fig:correlationMatricesBackground}
\end{figure}

\input{Tables/ranking_barrel.txt}

\bibliographystyle{plain}
\bibliography{BsMuMuCrossCheckAnalysis}

\end{document}
